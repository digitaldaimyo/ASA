\section{Method}

\subsection{Addressed State Attention}

At each layer, \ASA maintains a set of per-head slot states. A forward pass
consists of four conceptual stages:

\begin{enumerate}
  \item \textbf{Write path:} token keys attend over learned slot keys, producing
  write weights.
  \item \textbf{Slot state update:} slot states aggregate token values using
  these write weights. Slot states persist across sequence positions, allowing
  information to accumulate.
  \item \textbf{Read path:} token queries attend over slot states to produce read
  vectors.
  \item \textbf{Output projection:} read vectors are projected back into token
  space and combined with the residual stream.
\end{enumerate}

Crucially, tokens never attend to other tokens directly. All information flow
is mediated through the slot states. The mechanism is inspired by persistent
slot-based attention and latent-array architectures while focusing on explicit
routing control \cite{locatello2020,jaegle2021,jaegle2021io}.

\subsection{Optional components in the public implementation}

The released implementation exposes several optional components that were used
in exploratory research and are included for completeness and
reproducibility:

\begin{itemize}
  \item \textbf{Content read path:} a parallel token-to-token attention path
  blended with slot reads via a learned gate. This allows partial fallback to
  conventional attention behavior.
  \item \textbf{Slotspace refinement:} slot-to-slot attention that allows slot
  states to exchange information, controlled by a learned scalar gate.
\end{itemize}

These components are extensions to the core \ASA primitive, which remains the
write $\rightarrow$ slot $\rightarrow$ read loop.

\subsection{\ASA as a control system}

While \ASA can be described as a memory mechanism, our experiments support a
stronger interpretation: slot routing functions as a control system over
content flow.

Across layers and sequence positions, routing states exhibit:

\begin{itemize}
  \item \textbf{Low intrinsic dimensionality:} late-window routing trajectories
  are often dominated by 1--3 principal components.
  \item \textbf{Longer timescales:} routing states evolve more slowly than token
  representations and persist across positions.
  \item \textbf{Entropy reduction across depth:} routing distributions become
  more committed in mid-to-late layers.
  \item \textbf{Coherent control fields:} gradients of factual logits align
  consistently with routing directions.
\end{itemize}

In contrast, residual hidden states exhibit higher dimensionality and weaker
causal influence once routing is fixed. These observations are characteristic
of a control manifold rather than a static memory lookup.
