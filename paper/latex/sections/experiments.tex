\section{Experiments}

The experiments in the notebooks are exploratory and heterogeneous. The public
release focuses on representative, minimal versions of each analysis for
clarity and reproducibility.

\subsection{Sanity and reproducibility tests}

We provide CPU-friendly tests that validate:

\begin{itemize}
  \item Forward and backward passes
  \item Shape invariants
  \item Non-degenerate gradients
  \item Determinism under fixed seeds
  \item Masking behavior
  \item Routing override hooks
  \item Intervention toggles
\end{itemize}

These tests serve as executable documentation of the \ASA control surface and
are intended to run in constrained environments (e.g., Colab CPU).

\subsection{Factual probe: Paris-margin}

To study factual recall, we use a simple contrastive metric evaluated after
prompts referencing France. This probe is not intended as a benchmark. Instead,
it provides a scalar signal that is sensitive to factual correctness and
amenable to controlled intervention.

\subsection{Routing geometry across layers}

We analyze late-window routing states using PCA and clustering:

\begin{itemize}
  \item Early layers exhibit diffuse, rotating routing mass.
  \item Mid layers show entropy reduction and partial alignment.
  \item Late layers exhibit coherent, low-dimensional structure despite reduced
  explained variance.
\end{itemize}

This indicates increasing commitment and specialization of routing behavior
across depth. \Cref{fig:timescales-vs-layer,fig:half-life-heatmap,fig:ess-heatmap}
show the canonical visualizations for routing timescales and persistence.

\begin{figure}[t]
  \centering
  \includegraphics[width=0.85\linewidth]{timescales_vs_layer.png}
  \caption{Routing timescales versus layer depth.}
  \label{fig:timescales-vs-layer}
\end{figure}

\begin{figure}[t]
  \centering
  \includegraphics[width=0.85\linewidth]{half_life_heatmap.png}
  \caption{Half-life heatmap for routing persistence across layers and heads.}
  \label{fig:half-life-heatmap}
\end{figure}

\begin{figure}[t]
  \centering
  \includegraphics[width=0.85\linewidth]{ess_heatmap.png}
  \caption{Effective state size (ESS) heatmap summarizing routing concentration.}
  \label{fig:ess-heatmap}
\end{figure}

\subsection{Finite-difference Jacobian analysis}

We estimate gradients of the Paris-margin with respect to:

\begin{itemize}
  \item Routing logits (projected into routing PCA space)
  \item Residual hidden states (projected into residual PCA space)
\end{itemize}

Key findings:

\begin{itemize}
  \item Early layers: gradients align strongly with the dominant routing PC.
  \item Mid layers: gradients rotate across routing dimensions.
  \item Late layers: gradients spread but remain coherent within routing space.
  \item Residual gradients show substantially weaker and less consistent effects.
\end{itemize}

\subsection{Matched-effect interventions: routing vs. residual}

To directly compare causal efficiency, we perform matched interventions:

\begin{itemize}
  \item \textbf{Routing intervention:} override routing weights along an
  estimated control direction.
  \item \textbf{Residual intervention:} inject perturbations into hidden states
  while freezing routing to baseline.
\end{itemize}

Results:

\begin{itemize}
  \item Routing perturbations produce monotonic, linear changes in factual logits.
  \item Residual perturbations produce near-zero effect under frozen routing.
  \item The disparity grows in later layers.
\end{itemize}

This establishes routing as the dominant causal pathway for factual recall in
\ASA-based models. \Cref{fig:second-order-scatter} provides the canonical
comparison visualization.

\begin{figure}[t]
  \centering
  \includegraphics[width=0.85\linewidth]{second_order_scatter.png}
  \caption{Routing versus residual intervention effects under matched magnitude.}
  \label{fig:second-order-scatter}
\end{figure}
